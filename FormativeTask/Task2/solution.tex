\documentclass[a4paper, oneside]{article}
\usepackage{amsmath}
\usepackage{xspace}
\usepackage{tabularx}
\usepackage{url}

\usepackage{enumitem}
\usepackage{graphicx}
\usepackage{subfigure}
\usepackage{float}

\usepackage[margin=3cm]{geometry}

\usepackage{etoolbox}

\usepackage{hyperref}
\usepackage{cleveref}
\usepackage{booktabs}

\newcommand\st{\textsuperscript{st}\xspace}
\newcommand\nd{\textsuperscript{nd}\xspace}
\newcommand\rd{\textsuperscript{rd}\xspace}
\newcommand\nth{\textsuperscript{th}\xspace} %\th is taken already

\newtoggle{outputAnswers}

% Toggle whether or not to include answers to homework questions.
 \toggletrue{outputAnswers}
%\togglefalse{outputAnswers}


\newcommand{\assignment}[0]{assignment\xspace}
\newcommand{\assignments}[0]{assignments\xspace}
\newcommand{\Assignment}[0]{Assignment\xspace}
\begin{document}
	\title{Computer Vision \& Imaging/ Robot Vision - Formative task}
	\date{\today}
	
	\maketitle
% 	\begin{tabularx}{\textwidth}{lX}
% 		Submission deadline: & 23:59 on the 2\nd of March\\
% 		Instructor: & Dr. Hyung Jin Chang\\
% 		Teaching Assistant: & Nora Horanyi\\		
% 		Contribution to overall module mark: & 10\%\\

% 		Submission Method: & This \assignment must be submitted through Canvas.\\
% 		\addlinespace[1cm]
% 		Name: & your name\\
% 		Username: & your username (abc123)\\		
% 		Study program: &  MSc your program\\
% 	\end{tabularx}

	\section*{Part 1}
	For this task, you will have to submit the following files:
    \begin{itemize}
    \item code \textbf{username\_formativetask2.m}
    \item Figures \textbf{username\_formativetask2_fig.pdf}
    
    \item[Question 1.1] \emph{Read the imageDir and create an appropriate datastore of all the images in the directory, justify your choice of datastore type. Convert all the images to grayscale and display the resulting images.}
    \\
    \\
    \\
	\item[Question 1.2] \emph{Load the camera parameters (cameraParams.mat) from imagedir. Open cameraParams.mat, you should now be able to see a number of parameters which you will need for the rest of the task}
\\
\\
\\
	\item[Question 1.3] \emph{Extract key common features from the images and estimate the camera position relative to the previous image. These changes in position should then be related in a global coordinate system, use bundle adjustment to refine all points. If you get stuck the script helperEstimateRelativePose can be used to relate the relative positions.}
	\\
	\\
	\\
	\item[Question 1.4] \emph{Plot the relative camera positions and 3-D features you have calculated, ensuring your graph is fully labelled and titled.}
    \end{itemize}

% 	\section*{Part 2}
% 	For this task, you will have to submit the following file:
%     \begin{itemize}
%     \item code \textbf{username\_assignment3\_part2.m}
%     \item binary edge images as \textbf{username\_assignment3\_chessboard\_edge.jpg},\\ and  \textbf{username\_assignment3\_maze\_edge.jpg}
%     \item output images as \textbf{username\_assignment3\_chessboard\_lines.jpg},\\ and  \textbf{username\_assignment3\_maze\_lines.jpg}
%     \end{itemize}

% 	\section*{Part 3}
% 	For this task, you will have to submit the following file:
%     \begin{itemize}
%     \item code \textbf{username\_assignment3\_part3.m}
%     \end{itemize}
    
\end{document}

