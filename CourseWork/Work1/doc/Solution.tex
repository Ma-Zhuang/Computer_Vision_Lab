\documentclass[a4paper, oneside]{article}
\usepackage{amsmath}
\usepackage{xspace}
\usepackage{tabularx}
\usepackage{url}

\usepackage{enumitem}
\usepackage{graphicx}
\usepackage{subfigure}
\usepackage{float}

\usepackage[margin=3cm]{geometry}

\usepackage{etoolbox}
\usepackage{cleveref}
\usepackage{hyperref}
\usepackage{booktabs}

\newcommand\st{\textsuperscript{st}\xspace}
\newcommand\nd{\textsuperscript{nd}\xspace}
\newcommand\rd{\textsuperscript{rd}\xspace}
\newcommand\nth{\textsuperscript{th}\xspace} %\th is taken already

\newtoggle{outputAnswers}

% Toggle whether or not to include answers to homework questions.
 \toggletrue{outputAnswers}
%\togglefalse{outputAnswers}


\newcommand{\assignment}[0]{assignment\xspace}
\newcommand{\assignments}[0]{assignments\xspace}
\newcommand{\Assignment}[0]{Assignment\xspace}
\begin{document}
	\title{Computer Vision and Imaging (extended) [06-30241]\\ \Assignment 1}
	\date{9am, Monday, 8 March 2021}
	
	\maketitle
	\begin{tabularx}{\textwidth}{lX}
		Submission deadline: & \textbf{9am (GMT), Monday, 15 March 2021}\\
		Instructor: & Dr. Hyung Jin Chang\\
		            & Dr. Yixing Gao\\
		            & Dr. Mohan Sridharan\\
		            & Dr. Masoumeh Mansouri\\
		Total marks: & 100\\
		Contribution to overall module mark: & 25\%\\
		Submission Method: & This \assignment must be submitted through Canvas.\\

		\addlinespace[1cm]
		Name: & Your name\\
		Username: & Your username (abc123)\\		
		Study program: &  BSc your program\\
	\end{tabularx}
	\section*{Part 1}
	\begin{description}		
		\item[Question 1.1 ] \emph{What affect does this kernel have? (Consider both this image and larger example images)}
		\\ Your answer
		\item[Question 1.2 ] \emph{What are their responses to uniform brightness? Are the filters isotropic or anisotropic and explain the terms. Which filter would you use for finding edges? What kind of edges would you use it to find?}
		\\ Your answer
		\item[Question 1.3]
		\emph{Is it possible to use the intrinsic and extrinsic camera properties to transform from a point in the camera 2D pixel coordinate system to a point in the world 3D coordinate system, and why?}
		\\ Your answer
				\item[Question 1.4]
		\emph{Taking a 1$cm^2$ 6 MegaPixel imaging array as an example, what is the limiting factor to increasing the number of sensors to increase the resolution of the image captured?}
		\\ Your answer

	\end{description}
	
	\section*{Part 2}
	For this task, you will have to submit the following file:
    \begin{itemize}
    \item code \textbf{username\_assignment1\_part2.m}
    \end{itemize}
\begin{description}
    \item [Question 2.1.1]
    \emph{Discuss your observations.}
    \\ Your answer
    \item [Question 2.1.2]
    \emph{How well did they work and why? Discuss your observations.}
    \\ Your answer
    \item [Question 2.2]
    \emph{Discuss your observations.}
    \\ Your answer    
    \item [Question 2.3]
    \emph{Perform image registration and display your result.}
    \\ Your answer
\end{description}

	\section*{Part 3}
	For this task, you will have to submit the following file:
    \begin{itemize}
    \item code \textbf{username\_assignment1\_part3.m}
    \end{itemize}

	\begin{description}	
    \item[Question 3.1 ] \emph{Compare your results to the results of using the OTSU's method and
comment on the reasons for the observed differences in results. If other methods produce worse results than the default OTSU's method, then please explain why you think this is.} \\ Your answer
    \item[Question 3.2 ] \emph{Explain the purpose
and the outcome(s) of any operation that you have used.} \\ Your answer
    
    \item[Question 3.3.1] \emph{Area (in pixels) of each cell}
    \\ Your answer
    \item[Question 3.3.2] \emph{Mean brightness (in green channel) of each cell}
    \\ Your answer
    \item[Question 3.3.3] \emph{Mean and standard deviation for the area and brightness for all the cells in the image}    
    \\ Your answer
    
    \item[Question 3.4] \emph{Repeat Questions 3.1, 3.2, and 3.3 on the image of E-coli.}
    \\ Your answer
    
	\end{description}
	
\end{document}